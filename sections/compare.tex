\section{Сравнительный анализ КТ и МРТ}
Компьютерная томография и магнитно-резонансная томография являются одними из наиболее значимых технологий в медицинской визуализации, и их конструктивные элементы напрямую отражают физические принципы, лежащие в основе этих методов.

\textbf{Компьютерная томография} основывается на рентгеновском излучении, что требует особой конструкции, обеспечивающей генерацию рентгеновских лучей, их детектирование и последующую обработку для построения изображений. Центральным компонентом КТ-аппарата является рентгеновская трубка, внутри которой находятся анод и катод. Под высоким напряжением электроны катода ускоряются и направляются к аноду, изготовленному из вольфрама, где при столкновении их энергия преобразуется в рентгеновское излучение. Большая часть этой энергии превращается в тепло, что требует применения мощной системы охлаждения для анода. Энергия рентгеновских фотонов определяется напряжением на катоде и материалом анода, а сам рентгеновский пучок ослабляется при прохождении через ткани, в зависимости от их плотности и состава. Этот процесс формирования контраста и составляет основу изображения, на котором различаются плотные ткани, такие как кости, и мягкие ткани, поглощающие излучение в меньшей степени[1].

Для регистрации рентгеновских лучей КТ использует кольцо детекторов, которое окружает тело пациента. Современные системы включают несколько рядов детекторов, которые преобразуют энергию фотонов в электрические сигналы. Обычно используются ксеноновые камеры или люминесцентные кристаллы, работающие в сочетании с фотоэлектронными умножителями. Совместно с детекторами система включает коллиматоры, ограничивающие пучок излучения и снижающие рассеивание, что уменьшает облучение пациента и повышает контраст изображения. Коллиматоры, регулируемые диафрагмами, контролируют ширину пучка, формируя узкий веерообразный поток лучей, необходимый для создания срезов с высоким разрешением[1].

После регистрации данных КТ-сканером компьютер выполняет математические преобразования. Излучение, ослабленное при прохождении через тело, переводится в так называемые КТ-числа, которые соответствуют различным уровням плотности тканей в единицах Хаунсфилда. Эти данные обрабатываются через алгоритмы обратной проекции, основанные на преобразовании Радона, что позволяет построить поперечные срезы структуры тела. Современные мультидетекторные системы КТ также поддерживают спиральное сканирование, которое создаёт последовательные срезы в течение непрерывного движения рентгеновской трубки вокруг пациента и позволяет получать объёмные реконструкции органов и тканей[1].

\textbf{Магнитно-резонансная томография} работает на основе совершенно другого физического принципа — ядерного магнитного резонанса (ЯМР). В основе МРТ лежит взаимодействие атомных ядер с магнитными полями. Центральным элементом МР-томографа является главный магнит, который создаёт постоянное магнитное поле, обычно с напряжённостью от 1,5 до 3 Тесла. Под воздействием этого поля протоны водорода в теле пациента ориентируются в одном направлении и начинают прецессировать с определённой частотой, зависящей от силы магнитного поля. Магниты могут быть сверхпроводящими, что позволяет поддерживать стабильное магнитное поле с минимальными энергетическими потерями. Такие магниты требуют охлаждения до крайне низких температур, достигаемых при помощи жидкого гелия[1].

Для создания изображения в МРТ используются также градиентные катушки, которые накладывают на основное магнитное поле дополнительные поля, изменяющиеся по трём осям. Эти изменения позволяют чётко определять положение прецессирующих протонов в пространстве, что делает возможным построение послойных изображений тела. Градиенты магнитного поля позволяют избирательно выделять конкретные срезы тела, и поэтому могут быть настроены для получения изображений в различных плоскостях, что особенно полезно для трёхмерной визуализации[1].

Радиочастотные (РЧ) катушки, являющиеся неотъемлемой частью МРТ, испускают импульсы радиоволн, которые приводят протоны в возбуждённое состояние. После окончания действия импульса протоны возвращаются в начальное состояние, излучая при этом энергию, которая регистрируется катушками. Время, за которое протоны восстанавливаются, зависит от типа ткани, что позволяет различать ткани на основе времён T1- и T2-релаксации. Эти свойства обеспечивают уникальные возможности МРТ для визуализации мягких тканей, таких как мышцы, мозг и связки, поскольку различия в контрасте, получаемом при МРТ, зависят от различий плотности водорода и времени релаксации в разных тканях[1].

Компьютерные системы, применяемые в МРТ, обрабатывают собранные данные, используя преобразования Фурье для анализа сигналов из разных областей пространства. Результат этих расчётов представляет собой изображения срезов тела, построенные в различных проекциях и плоскостях, что даёт возможность получать детализированные изображения сложных структур организма без ионизирующего излучения. МРТ позволяет с высокой точностью исследовать мягкотканные структуры и является предпочтительным методом для диагностики в неврологии и онкологии[1].

Таким образом, конструкции КТ и МРТ сканеров отличаются в зависимости от физических принципов, лежащих в основе их работы. Компьютерная томография позволяет получать высококачественные изображения твёрдых тканей и обладает высокой скоростью сканирования, тогда как магнитно-резонансная томография обеспечивает превосходный контраст мягких тканей и является безопасной за счёт отсутствия ионизирующего излучения.
