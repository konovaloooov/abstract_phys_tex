В ходе работы мы познакомились с такими Технологиями, как КТ и МРТ. Это весьма сложные, но интересные с точки зрения физики явления. В наше время компьютерная и магнитно-резонансная томография стала доступной диагностической процедурой для всех, но далеко не каждый знаком с принципом работы данных технологий.

Компьютерная томография (КТ) — это передовой метод визуализации, позволяющий получать послойные изображения организма с помощью рентгеновского излучения. В отличие от традиционной рентгенографии, КТ позволяет исследовать тело посрезово, обеспечивая высокую детализацию анатомических структур. 

Магнитно-резонансная томография (МРТ) — метод визуализации, основанный на ядерном магнитном резонансе (ЯМР). ЯМР наблюдается, когда ядра с ненулевым спином, такие как протоны водорода,поглощают радиочастотные волны при определённых условиях, что вызывает изменение ориентации их магнитных моментов. 

При сравнительном анализе стало ясно, что это абсолютно разные технологии, каждая из которых применяется врачами для своих целей.  конструкции КТ и МРТ сканеров отличаются в зависимости от физических принципов, лежащих в основе их работы. Компьютерная томография позволяет получать высококачественные изображения твёрдых тканей и обладает высокой скоростью сканирования, тогда как магнитно-резонансная томография обеспечивает превосходный контраст мягких тканей и является безопасной за счёт отсутствия ионизирующего излучения.