\section{История создания и важные достижения}
\subsection{Открытие рентгеновских лучей и их применение в медицине (КТ)}
Открытие рентгеновских лучей Вильгельмом Рентгеном в 1895 году совершило революцию в медицинской диагностике. Используя катодные лучи, он впервые наблюдал новое излучение, которое могло проникать через различные вещества, включая ткани человека, и создавать на фотопластинке теневое изображение плотных структур, таких как кости. Этот эффект, основанный на разности в поглощении рентгеновского излучения различными материалами, стал основой рентгенографии и заложил базу для более сложных методов, таких как компьютерная томография (КТ)\cite{history}.

Рентгеновские лучи представляют собой высокоэнергетические электромагнитные волны с длиной волны от 0,01 до 10 нанометров, что позволяет им проходить через мягкие ткани организма, поглощаясь при этом более плотными структурами. Это явление стало основой для дифференциации тканей на рентгеновском изображении, где кости выглядят более светлыми, а мягкие ткани более тёмными\cite{history}.

С появлением компьютерной томографии (КТ) в 1970-х годах открылись новые возможности. Метод КТ использует принцип послойного сканирования организма с помощью множества рентгеновских проекций под разными углами, что позволяет получать трёхмерные изображения. Этот метод разрабатывался физиками Алланом Кормаком и Годфри Хаунсфилдом, которые внесли решающий вклад в алгоритмы реконструкции изображений. Хаунсфилд впервые применил технологию КТ для медицинской диагностики, сосредоточив внимание на исследованиях головного мозга и черепа\cite{history}.
Так, самые первые аппараты первого поколения, которые появились в 1973 году, были пошаговыми. В томографе была всего одна рентгеновская трубка, которая была направлена на один детектор. Один оборот позволял получить изображение одного слоя. Во втором поколении томографов за основу брался веерный тип конструкции, при котором напротив трубки устанавливалось несколько детекторов. Время обработки изображения занимало не 4-5 минут, как в случае аппаратов первого поколения, а значительно меньше - порядка 20 секунд\cite{one}.

Третье поколение КТ-аппаратов ввело термин спиральной компьютерной томографии. История спиральной КТ берет свое начало с 1988 года, когда компанией Siemens Medical Solutions был предложен первый спиральный томограф. Принцип работы аппарата основан на одновременном вращении рентгеновской трубки, которая генерирует излучение, и непрерывного движения стола, на котором лежит пациент, вокруг продольной оси сканирования. При такой комбинации траектория движения трубки относительно направления движения стола принимает форму спирали. Такая технология сделала возможным сократить время исследования и уменьшить лучевую нагрузку на организм пациента\cite{one}.

Несколькими годами позже, в 1992 году, компанией Elscint Co был предложен метод мультиспиральной КТ – МСКТ. Главным отличием такой томографии стало наличие не одного, а двух и более детекторов. В этом году был представлен первый двухсрезовый МСКТ томограф, обладающий двумя рядами детекторов, а в 1998 году - четырехсрезовые с 4-мя рядами соответственно. Помимо количества детекторов также было увеличено число оборотов трубки до двух раз в секунду, что сделало возможным еще больше снизить время обследования и повысить качество изображения. Метод МСКТ стал стремительно развиваться, и в начале ХХI века, в 2004-2005 гг. были представлены 32-, 64-, 128- срезовые томографы\cite{one}. 
%(mrt-kt.ru/stati/istoria_kt)

В 2007 году компанией Toshiba были сконструированы 320-срезовые МСКТ-томографы, которые стали новым этапом развития метода КТ. Такое оборудование позволяет не только получать высокоинформативные изображения, но и буквально в реальном времени наблюдать за процессами, происходящими в сердце и головном мозге. МСКТ помимо уменьшения времени и лучевой нагрузки на пациента имеет ряд преимуществ перед методом спиральной КТ: увеличение зоны анатомического покрытия, скорости сканирования, отношения сигнал/шум, улучшение контрастного разрешения\cite{one}. 
%(https://mrt-kt.ru/stati/istoria_kt)

 Компьютерная томография привела к важным улучшениям диагностики благодаря возможности получения срезов тканей, что помогло значительно повысить разрешение и детализацию изображений. Использование метода компьютерной томографии совершило революционный переворот в диагностике, особенно при исследовании заболеваний нервной системы.  Этот метод значительно увеличил точность диагностики и стал особенно важен для исследования сложных анатомических структур, таких как головной мозг и органы грудной клетки\cite{history}.

 \subsection{Развитие теории магнитного резонанса и создание МРТ}

В 1945 году две группы физиков, работающих независимо друг от друга "--- Пурселл Ричард, Тори и Паунд в Гарвардском университете, а Блох Феликс, Хансен и Паккард "--- в Стандфорском, впервые успешно наблюдали явление ядерно-магнитного резонанса в твердых телах и жидкостях.  Их усилиями были заложены основы теории магнитного резонанса. В своих экспериментах они продемонстрировали, что атомные ядра, помещенные в сильное магнитное поле, могут поглощать и затем излучать радиоволны. Этот феномен получил название ядерного магнитного резонанса (ЯМР). За открытие феномена ядерно-магнитного резонанса Пурселл и Блох были удостоены Нобелевской премии в 1952 г., что подчёркивает значимость этого открытия для физики и медицины\cite{history}.

Для медицины это стало поворотным моментом, так как на основе ЯМР удалось разработать метод получения изображений мягких тканей, что ранее было невозможно с использованием рентгеновских лучей. "Годом основания магнитно-резонансной томографии принято считать 1973 год, когда профессор химии и радиологии из Нью-Йоркского университета Стони Брук — Пол Лотербур, опубликовал в журнале Nature статью «Создание изображения с помощью индуцированного локального взаимодействия: примеры на основе магнитного резонанса», в которой были представлены трехмерные изображения объектов, полученные по спектрам протонного магнитного резонанса воды из этих объектов. Эта работа и легла в основу метода магнитной резонансной томографии (МРТ). Позже доктор Питер Мэнсфилд усовершенствовал математические алгоритмы получения изображения\cite{two}.

Лаутербур и Мэнсфилд были награждены Нобелевской премией в 2003 году за их вклад в разработку МРТ", подчёркивается важность их работы для медицины и науки\cite{history}.

В 1980 году Эдельштейн с сотрудниками, используя этот метод, продемонстрировали отображение человеческого тела. Для получения одного изображения требовалось приблизительно 5 минут. К 1986 году время отображения было снижено до 5 секунд без какой-либо значимой потери качества. В том же году был создан ЯМР-микроскоп, который позволял добиваться разрешения 10 mм на образцах размером в 1 см. В 1988 году Думоулин усовершенствовал МР - ангиографию, которая делала возможным отображение текущей крови без применения контрастных агентов. В 1989 году был представлен метод планарной томографии, который позволял захватывать изображения с видеочастотами (30 мс)\cite{two}.

МРТ позволяет получать изображения с высокой контрастностью между различными типами мягких тканей, таких как мышцы, связки и головной мозг. Это стало возможным благодаря двум основным процессам релаксации — T1 и T2, которые происходят после воздействия радиочастотного импульса на ядра водорода. T1-релаксация связана с восстановлением продольного магнитного вектора, а T2-релаксация — с потерей фазового согласования поперечных компонент магнитного вектора. Эти процессы зависят от состава и плотности тканей, что делает МРТ чувствительным методом для исследования анатомии и патологии различных органов\cite{history}.

Современные МРТ-аппараты обеспечивают возможность получения изображений с высоким разрешением и контрастом, что особенно полезно для диагностики заболеваний центральной нервной системы и других мягкотканных структур. МРТ продолжает развиваться и играет ключевую роль в медицинской практике, обеспечивая точность и неинвазивность диагностики\cite{history}.
