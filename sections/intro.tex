Компьютерная томография и магнитно-резонансная томография являются одними из наиболее значимых технологий в медицинской визуализации, и их конструктивные элементы напрямую отражают физические принципы, лежащие в основе этих методов.

В наше время компьютерная и магнитно-резонансная томография стала доступной диагностической процедурой для всех, и очень популярной среди врачей. Часто томография является единственным диагностическим методом для подтверждения диагноза, но также используется для  его уточнения. Именно поэтому КТ и МРТ является актуальной и очень важной диагностической процедурой как для врачей, так и для пациентов.

Обычному человеку кажется, что это два почти одинаковых способа выявления различных диагнозов, но это совсем не так. В своей работе я хочу более детально с физической точки зрения познакомиться с каждым методом диагностики, а также провести сравненительный анализ. Для начала стоит ознакомиться с историей развития данных технологий и узнать самые важные открытия.