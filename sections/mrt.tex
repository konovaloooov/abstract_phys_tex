\section{Физические основы магнитно-резонансной томографии}

Магнитно-резонансная томография (МРТ) "--- метод визуализации, основанный на ядерном магнитном резонансе (ЯМР). ЯМР наблюдается, когда ядра с ненулевым спином(собственный момент количества движения (момент импульса) элементарных частиц, имеющий квантовую природу и не связанный с их перемещением в пространстве как целого), такие как протоны водорода, поглощают радиочастотные волны при определённых условиях, что вызывает изменение ориентации их магнитных моментов. Протоны водорода особенно полезны для МРТ из-за их обилия в организме в виде молекул воды и жира. ЯМР является результатом взаимодействия между внешним магнитным полем и магнитным моментом ядер, которые ориентируются вдоль или против поля, создавая два возможных энергетических уровня\cite{ktmrt}.

\begin{figure}[H]
    \centering
    \includegraphics[width=0.7\textwidth]{pic/8.png}
    \caption{Схема МР-томографа}
    \label{fig:image1}
\end{figure}

Процесс, при котором протоны переходят из одного энергетического уровня в другой при воздействии магнитного поля, объясняется понятием ларморовской прецессии.
\textbf{Ларморовская прецессия} "--- прецессия (вращение как целого) магнитного момента электронов, атомного ядра и атомов вокруг вектора внешнего магнитного поля. Ларморовская частота, определяющая частоту прецессии в зависимости от напряжённости магнитного поля, является ключевой для настройки радиочастотных импульсов. Она рассчитывается по формуле:

\[
\omega = \gamma B_0,
\]

где \(\omega\) — частота прецессии, \(\gamma\) — гиромагнитное отношение (специфическое для каждого типа ядер), \(B_0\) — напряжённость магнитного поля. 

Для протонов водорода гиромагнитное отношение составляет около 42,58 МГц/Тл, что позволяет использовать поля в диапазоне 1,5–3 Тл в клинической практике\cite{ktmrt}.

\subsection{Взаимодействие протонов водорода с внешним магнитным полем}

Для создания изображения МРТ используется взаимодействие протонов водорода, которые составляют значительную часть тканей организма, с внешним магнитным полем. В обычных условиях магнитные моменты протонов хаотически ориентированы, но при наличии магнитного поля они стремятся выстроиться вдоль или против направления поля. Это приводит к появлению макроскопической намагниченности вдоль оси поля, создавая измеримый магнитный момент, который становится основой для регистрации сигнала\cite{ktmrt}.
\begin{figure}[H]
    \centering
    \includegraphics[width=0.7\textwidth]{pic/7.png}
    \caption{Принцип приема и передачи сигнала в МРТ — томографе}
    \label{fig:image1}
\end{figure}
Энергетическое распределение магнитных моментов протонов между двумя состояниями можно объяснить законом распределения Больцмана, который говорит, что количество протонов, ориентированных по полю, немного больше тех, что против поля. Эта разница в числе протонов создаёт слабую намагниченность, которую фиксирует аппарат. В условиях слабого магнитного поля эта намагниченность незначительна, но при высоких полях, как в МРТ, она становится достаточно большой для создания диагностически значимого сигнала\cite{ktmrt}.

\subsection{Влияние радиочастотного импульса и формирование сигнала}

Для создания контрастного изображения протоны возбуждаются с помощью радиочастотного импульса, который заставляет их переходить на более высокий энергетический уровень. После того как РЧ-импульс отключается, протоны возвращаются в исходное состояние, испуская энергию в виде сигнала, который улавливается приёмником МРТ. Процесс возврата протонов в исходное состояние называется релаксацией и делится на два типа: продольную (T1) и поперечную (T2) релаксацию, каждый из которых имеет диагностическое значение\cite{ktmrt}.

Продольная релаксация (T1) характеризует скорость восстановления намагниченности вдоль оси внешнего поля, а поперечная релаксация (T2) "--- скорость потери намагниченности в поперечной плоскости. Эти процессы описываются формулами:

Для продольной релаксации:
\[
M_z(t) = M_0 \left(1 - e^{-\frac{t}{T1}}\right),
\]

где \(M_z\) — продольная намагниченность в момент времени \(t\), \(M_0\) — максимальная намагниченность, \(T1\) — время продольной релаксации. 

Этот процесс зависит от свойств тканей, поэтому различные ткани имеют разные значения \(T1\), что позволяет различать их на изображении\cite{ktmrt}.

Для поперечной релаксации:
\[
M_{xy}(t) = M_{xy}(0) e^{-\frac{t}{T2}},
\]

где \(M_{xy}\) — поперечная намагниченность, \(T2\) — время, за которое сигнал ослабевает из-за взаимодействий между спинами. 

Различия в T2-релаксации дают информацию о физических и химических различиях между тканями, что обеспечивает контраст на изображениях\cite{ktmrt}.

\begin{figure}[H]
    \centering
    \includegraphics[width=0.6\textwidth]{pic/9.png}
    \caption{Слева направо Т 1-взвешенное, Т 2-взвешенное и взвешенное по протонной плотности изображения}
    \label{fig:image1}
\end{figure}

\subsection{Использование градиентов магнитного поля для локализации сигнала}

Для получения точного изображения используются градиенты магнитного поля(изменение магнитного поля в зависимости от положения), которые обеспечивают пространственную локализацию сигнала. Градиенты создают изменяющуюся частоту прецессии по оси X, Y и Z, что позволяет различать сигналы от разных точек тела. Градиенты, изменяя частоту Лармора в зависимости от положения, позволяют сфокусироваться на определённых срезах, что значительно улучшает пространственное разрешение. Путём применения градиентов можно выбирать толщину среза, направление и масштаб изображения\cite{ktmrt}.

\subsection{Типы последовательностей и качество изображения}

Качество изображения в МРТ зависит от выбранных последовательностей импульсов, которые позволяют настраивать параметры съёмки для разных типов тканей и целей исследования. Существует несколько типов последовательностей, таких как спиновое эхо и градиентное эхо, которые используются для оптимизации контраста и разрешения. Последовательности спинового эха обеспечивают более чёткие изображения за счёт компенсации неоднородностей магнитного поля, тогда как градиентное эхо позволяет получать изображения с высоким разрешением за меньшее время\cite{ktmrt}.

Последовательности спинового эха используются для повышения точности при изучении мягких тканей, таких как головной мозг и внутренние органы, а градиентное эхо — для быстрого сканирования и оценки движения, например, при кардиологических исследованиях\cite{ktmrt}.